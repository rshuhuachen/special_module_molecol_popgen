% Options for packages loaded elsewhere
\PassOptionsToPackage{unicode}{hyperref}
\PassOptionsToPackage{hyphens}{url}
\PassOptionsToPackage{dvipsnames,svgnames,x11names}{xcolor}
%
\documentclass[
  letterpaper,
  DIV=11,
  numbers=noendperiod]{scrreprt}

\usepackage{amsmath,amssymb}
\usepackage{iftex}
\ifPDFTeX
  \usepackage[T1]{fontenc}
  \usepackage[utf8]{inputenc}
  \usepackage{textcomp} % provide euro and other symbols
\else % if luatex or xetex
  \usepackage{unicode-math}
  \defaultfontfeatures{Scale=MatchLowercase}
  \defaultfontfeatures[\rmfamily]{Ligatures=TeX,Scale=1}
\fi
\usepackage{lmodern}
\ifPDFTeX\else  
    % xetex/luatex font selection
\fi
% Use upquote if available, for straight quotes in verbatim environments
\IfFileExists{upquote.sty}{\usepackage{upquote}}{}
\IfFileExists{microtype.sty}{% use microtype if available
  \usepackage[]{microtype}
  \UseMicrotypeSet[protrusion]{basicmath} % disable protrusion for tt fonts
}{}
\makeatletter
\@ifundefined{KOMAClassName}{% if non-KOMA class
  \IfFileExists{parskip.sty}{%
    \usepackage{parskip}
  }{% else
    \setlength{\parindent}{0pt}
    \setlength{\parskip}{6pt plus 2pt minus 1pt}}
}{% if KOMA class
  \KOMAoptions{parskip=half}}
\makeatother
\usepackage{xcolor}
\setlength{\emergencystretch}{3em} % prevent overfull lines
\setcounter{secnumdepth}{5}
% Make \paragraph and \subparagraph free-standing
\makeatletter
\ifx\paragraph\undefined\else
  \let\oldparagraph\paragraph
  \renewcommand{\paragraph}{
    \@ifstar
      \xxxParagraphStar
      \xxxParagraphNoStar
  }
  \newcommand{\xxxParagraphStar}[1]{\oldparagraph*{#1}\mbox{}}
  \newcommand{\xxxParagraphNoStar}[1]{\oldparagraph{#1}\mbox{}}
\fi
\ifx\subparagraph\undefined\else
  \let\oldsubparagraph\subparagraph
  \renewcommand{\subparagraph}{
    \@ifstar
      \xxxSubParagraphStar
      \xxxSubParagraphNoStar
  }
  \newcommand{\xxxSubParagraphStar}[1]{\oldsubparagraph*{#1}\mbox{}}
  \newcommand{\xxxSubParagraphNoStar}[1]{\oldsubparagraph{#1}\mbox{}}
\fi
\makeatother

\usepackage{color}
\usepackage{fancyvrb}
\newcommand{\VerbBar}{|}
\newcommand{\VERB}{\Verb[commandchars=\\\{\}]}
\DefineVerbatimEnvironment{Highlighting}{Verbatim}{commandchars=\\\{\}}
% Add ',fontsize=\small' for more characters per line
\newenvironment{Shaded}{}{}
\newcommand{\AlertTok}[1]{\textcolor[rgb]{1.00,0.33,0.33}{\textbf{#1}}}
\newcommand{\AnnotationTok}[1]{\textcolor[rgb]{0.42,0.45,0.49}{#1}}
\newcommand{\AttributeTok}[1]{\textcolor[rgb]{0.84,0.23,0.29}{#1}}
\newcommand{\BaseNTok}[1]{\textcolor[rgb]{0.00,0.36,0.77}{#1}}
\newcommand{\BuiltInTok}[1]{\textcolor[rgb]{0.84,0.23,0.29}{#1}}
\newcommand{\CharTok}[1]{\textcolor[rgb]{0.01,0.18,0.38}{#1}}
\newcommand{\CommentTok}[1]{\textcolor[rgb]{0.42,0.45,0.49}{#1}}
\newcommand{\CommentVarTok}[1]{\textcolor[rgb]{0.42,0.45,0.49}{#1}}
\newcommand{\ConstantTok}[1]{\textcolor[rgb]{0.00,0.36,0.77}{#1}}
\newcommand{\ControlFlowTok}[1]{\textcolor[rgb]{0.84,0.23,0.29}{#1}}
\newcommand{\DataTypeTok}[1]{\textcolor[rgb]{0.84,0.23,0.29}{#1}}
\newcommand{\DecValTok}[1]{\textcolor[rgb]{0.00,0.36,0.77}{#1}}
\newcommand{\DocumentationTok}[1]{\textcolor[rgb]{0.42,0.45,0.49}{#1}}
\newcommand{\ErrorTok}[1]{\textcolor[rgb]{1.00,0.33,0.33}{\underline{#1}}}
\newcommand{\ExtensionTok}[1]{\textcolor[rgb]{0.84,0.23,0.29}{\textbf{#1}}}
\newcommand{\FloatTok}[1]{\textcolor[rgb]{0.00,0.36,0.77}{#1}}
\newcommand{\FunctionTok}[1]{\textcolor[rgb]{0.44,0.26,0.76}{#1}}
\newcommand{\ImportTok}[1]{\textcolor[rgb]{0.01,0.18,0.38}{#1}}
\newcommand{\InformationTok}[1]{\textcolor[rgb]{0.42,0.45,0.49}{#1}}
\newcommand{\KeywordTok}[1]{\textcolor[rgb]{0.84,0.23,0.29}{#1}}
\newcommand{\NormalTok}[1]{\textcolor[rgb]{0.14,0.16,0.18}{#1}}
\newcommand{\OperatorTok}[1]{\textcolor[rgb]{0.14,0.16,0.18}{#1}}
\newcommand{\OtherTok}[1]{\textcolor[rgb]{0.44,0.26,0.76}{#1}}
\newcommand{\PreprocessorTok}[1]{\textcolor[rgb]{0.84,0.23,0.29}{#1}}
\newcommand{\RegionMarkerTok}[1]{\textcolor[rgb]{0.42,0.45,0.49}{#1}}
\newcommand{\SpecialCharTok}[1]{\textcolor[rgb]{0.00,0.36,0.77}{#1}}
\newcommand{\SpecialStringTok}[1]{\textcolor[rgb]{0.01,0.18,0.38}{#1}}
\newcommand{\StringTok}[1]{\textcolor[rgb]{0.01,0.18,0.38}{#1}}
\newcommand{\VariableTok}[1]{\textcolor[rgb]{0.89,0.38,0.04}{#1}}
\newcommand{\VerbatimStringTok}[1]{\textcolor[rgb]{0.01,0.18,0.38}{#1}}
\newcommand{\WarningTok}[1]{\textcolor[rgb]{1.00,0.33,0.33}{#1}}

\providecommand{\tightlist}{%
  \setlength{\itemsep}{0pt}\setlength{\parskip}{0pt}}\usepackage{longtable,booktabs,array}
\usepackage{calc} % for calculating minipage widths
% Correct order of tables after \paragraph or \subparagraph
\usepackage{etoolbox}
\makeatletter
\patchcmd\longtable{\par}{\if@noskipsec\mbox{}\fi\par}{}{}
\makeatother
% Allow footnotes in longtable head/foot
\IfFileExists{footnotehyper.sty}{\usepackage{footnotehyper}}{\usepackage{footnote}}
\makesavenoteenv{longtable}
\usepackage{graphicx}
\makeatletter
\newsavebox\pandoc@box
\newcommand*\pandocbounded[1]{% scales image to fit in text height/width
  \sbox\pandoc@box{#1}%
  \Gscale@div\@tempa{\textheight}{\dimexpr\ht\pandoc@box+\dp\pandoc@box\relax}%
  \Gscale@div\@tempb{\linewidth}{\wd\pandoc@box}%
  \ifdim\@tempb\p@<\@tempa\p@\let\@tempa\@tempb\fi% select the smaller of both
  \ifdim\@tempa\p@<\p@\scalebox{\@tempa}{\usebox\pandoc@box}%
  \else\usebox{\pandoc@box}%
  \fi%
}
% Set default figure placement to htbp
\def\fps@figure{htbp}
\makeatother

\usepackage{fvextra}
\DefineVerbatimEnvironment{Highlighting}{Verbatim}{breaklines,commandchars=\\\{\}}
\DefineVerbatimEnvironment{OutputCode}{Verbatim}{breaklines,commandchars=\\\{\}}
\KOMAoption{captions}{tableheading}
\makeatletter
\@ifpackageloaded{tcolorbox}{}{\usepackage[skins,breakable]{tcolorbox}}
\@ifpackageloaded{fontawesome5}{}{\usepackage{fontawesome5}}
\definecolor{quarto-callout-color}{HTML}{909090}
\definecolor{quarto-callout-note-color}{HTML}{0758E5}
\definecolor{quarto-callout-important-color}{HTML}{CC1914}
\definecolor{quarto-callout-warning-color}{HTML}{EB9113}
\definecolor{quarto-callout-tip-color}{HTML}{00A047}
\definecolor{quarto-callout-caution-color}{HTML}{FC5300}
\definecolor{quarto-callout-color-frame}{HTML}{acacac}
\definecolor{quarto-callout-note-color-frame}{HTML}{4582ec}
\definecolor{quarto-callout-important-color-frame}{HTML}{d9534f}
\definecolor{quarto-callout-warning-color-frame}{HTML}{f0ad4e}
\definecolor{quarto-callout-tip-color-frame}{HTML}{02b875}
\definecolor{quarto-callout-caution-color-frame}{HTML}{fd7e14}
\makeatother
\makeatletter
\@ifpackageloaded{bookmark}{}{\usepackage{bookmark}}
\makeatother
\makeatletter
\@ifpackageloaded{caption}{}{\usepackage{caption}}
\AtBeginDocument{%
\ifdefined\contentsname
  \renewcommand*\contentsname{Table of contents}
\else
  \newcommand\contentsname{Table of contents}
\fi
\ifdefined\listfigurename
  \renewcommand*\listfigurename{List of Figures}
\else
  \newcommand\listfigurename{List of Figures}
\fi
\ifdefined\listtablename
  \renewcommand*\listtablename{List of Tables}
\else
  \newcommand\listtablename{List of Tables}
\fi
\ifdefined\figurename
  \renewcommand*\figurename{Figure}
\else
  \newcommand\figurename{Figure}
\fi
\ifdefined\tablename
  \renewcommand*\tablename{Table}
\else
  \newcommand\tablename{Table}
\fi
}
\@ifpackageloaded{float}{}{\usepackage{float}}
\floatstyle{ruled}
\@ifundefined{c@chapter}{\newfloat{codelisting}{h}{lop}}{\newfloat{codelisting}{h}{lop}[chapter]}
\floatname{codelisting}{Listing}
\newcommand*\listoflistings{\listof{codelisting}{List of Listings}}
\makeatother
\makeatletter
\makeatother
\makeatletter
\@ifpackageloaded{caption}{}{\usepackage{caption}}
\@ifpackageloaded{subcaption}{}{\usepackage{subcaption}}
\makeatother

\usepackage{bookmark}

\IfFileExists{xurl.sty}{\usepackage{xurl}}{} % add URL line breaks if available
\urlstyle{same} % disable monospaced font for URLs
\hypersetup{
  pdftitle={Special Module: Methods in Mol Ecol and Pop Gen},
  pdfauthor={Rebecca S. Chen},
  colorlinks=true,
  linkcolor={blue},
  filecolor={Maroon},
  citecolor={Blue},
  urlcolor={Blue},
  pdfcreator={LaTeX via pandoc}}


\title{Special Module: Methods in Mol Ecol and Pop Gen}
\author{Rebecca S. Chen}
\date{}

\begin{document}
\maketitle

\renewcommand*\contentsname{Table of contents}
{
\hypersetup{linkcolor=}
\setcounter{tocdepth}{1}
\tableofcontents
}

\bookmarksetup{startatroot}

\chapter{Methods in Mol Ecol and Pop
Gen}\label{methods-in-mol-ecol-and-pop-gen}

\bookmarksetup{startatroot}

\chapter{Introduction}\label{introduction}

This webpage/document contains all the practical materials for the
Spezial Modul entitled ``Methods in Molecular Ecology and Population
Genetics''.

The order of the practicals/tests on this webpage are in the order of
the way we will discuss them throughout the course. Please don't
hestitate to ask David or Rebecca any questions throughout the course.

If anything is unclear or not working, please let us know!

Each session will contain text, exercises and answers. We encourage you
to really think before you reveal the answer, otherwise you won't learn.
If you need any help getting to the answer, you can also ask us
questions to put you in the right direction. The same applies to helping
you troubleshoot, of course.

Remember, even the experts will google how to code things. Once you get
more fluent in the language, you can do simple things by heart. But just
like learning any langauge, this takes time and practise. So don't be
afraid to look into your notes or look up how to do things, this is how
every person works in R

\begin{figure}[H]

{\centering \pandocbounded{\includegraphics[keepaspectratio]{img/google.png}}

}

\caption{Use the resources available to you!}

\end{figure}%

\bookmarksetup{startatroot}

\chapter{Basic skills - a test}\label{basic-skills---a-test}

\section{Datacamp}\label{datacamp}

In the first week, you will have almost two full days to learn the
basics of R yourself. We have given you access to
\href{https://datacamp.com/}{DataCamp}, a great interactive platform to
learn R at any level. The courses we would like you to complete because
they prepare you with all you need to know to follow the rest of the
course are as follows:

\begin{itemize}
\tightlist
\item
  ``Introduction to R''
\item
  ``Introduction to tidyverse''
\item
  ``Introduction to dplyr''
\end{itemize}

Please follow them at your own speed. If you think something is too easy
because you already have some prior experience, you can skip through it
and learn at your own speed. Feel free to start other courses too. All
we want you to be able to do, is gain the skills and experience to be
able to complete the exercises below. These exercises are therefore not
graded in any way, it's just a way to make sure you understand the
basics to continue with the rest of the course.

\begin{tcolorbox}[enhanced jigsaw, breakable, title=\textcolor{quarto-callout-important-color}{\faExclamation}\hspace{0.5em}{A note on AI}, toptitle=1mm, leftrule=.75mm, toprule=.15mm, colbacktitle=quarto-callout-important-color!10!white, opacityback=0, opacitybacktitle=0.6, bottomtitle=1mm, arc=.35mm, titlerule=0mm, colframe=quarto-callout-important-color-frame, left=2mm, bottomrule=.15mm, rightrule=.15mm, coltitle=black, colback=white]

You can find the answers and code when you click on the toggle, but try
to do as much as you can without looking at the answers first. You can
use Google or your notes from DataCamp to find the answers, but please
don't use ChatGPT or other AI tools to help you find the answers
directly. For example, don't just put the question as a prompt, but you
could ask: how do I calculate the mean in R? Or: how can I make a basic
plot with ggplot2 to visualise XYZ? In other words: don't `cheat', but
if you want to use ChatGPT, use it in a way that will help you learn and
it doesn't hand the answer to you.

\end{tcolorbox}

\subsection{Data description}\label{data-description}

Let's work with some global height dataset. It was downloaded from the
website \url{https://ncdrisc.org/data-downloads-height.html} and the
file can be found in the \texttt{data} directory. The data contain the
mean heights of adolescents of both boys and girls at different age
groups. The mean height is the mean height of a certain age group in a
specific year.

\subsection{Part 1: exploration}\label{part-1-exploration}

\begin{itemize}
\tightlist
\item
  The dataset is called
  \texttt{NCD\_RisC\_Lancet\_2020\_heigh\_child\_adolescent\_global.csv}
  and is stored in the \texttt{data} directory. What does the
  \texttt{.csv} extension mean?
\end{itemize}

\begin{tcolorbox}[enhanced jigsaw, breakable, title=\textcolor{quarto-callout-tip-color}{\faLightbulb}\hspace{0.5em}{Answer}, toptitle=1mm, leftrule=.75mm, toprule=.15mm, colbacktitle=quarto-callout-tip-color!10!white, opacityback=0, opacitybacktitle=0.6, bottomtitle=1mm, arc=.35mm, titlerule=0mm, colframe=quarto-callout-tip-color-frame, left=2mm, bottomrule=.15mm, rightrule=.15mm, coltitle=black, colback=white]

Csv stands for comma separated file. It is a file format to store tables
(such as Excel files), where each line represents a data record, and
fields are separated by commas. CSV files are commonly used to store
data, but you have to be careful if your data contains text that might
include comma's, because it will think the sentence needs to be split up
into different fields! Other ways to separate fields are for example by
using tabs (\t) or spaces.

\end{tcolorbox}

\begin{itemize}
\tightlist
\item
  Import the data set and assign it to a variable (called
  e.g.~\texttt{data})\\
\end{itemize}

\begin{tcolorbox}[enhanced jigsaw, breakable, title=\textcolor{quarto-callout-tip-color}{\faLightbulb}\hspace{0.5em}{Code and answer}, toptitle=1mm, leftrule=.75mm, toprule=.15mm, colbacktitle=quarto-callout-tip-color!10!white, opacityback=0, opacitybacktitle=0.6, bottomtitle=1mm, arc=.35mm, titlerule=0mm, colframe=quarto-callout-tip-color-frame, left=2mm, bottomrule=.15mm, rightrule=.15mm, coltitle=black, colback=white]

We are working within an R project, so we can use relative path names.
If you prefer using the absolute path name, this might look different.
Also, path specification is dependent on your operating system. This has
been done in iOS (Apple), but Windows looks different.

\begin{Shaded}
\begin{Highlighting}[]
\CommentTok{\# using read.csv}
\NormalTok{data }\OtherTok{\textless{}{-}} \FunctionTok{read.csv}\NormalTok{(}\StringTok{"../data/NCD\_RisC\_Lancet\_2020\_height\_child\_adolescent\_global.csv"}\NormalTok{)}

\CommentTok{\# or absolute path name}
\NormalTok{data }\OtherTok{\textless{}{-}} \FunctionTok{read.csv}\NormalTok{(}\StringTok{"/Users/vistor/Documents/Work/GitHub/postdoc/special\_module\_molecol\_popgen/data/NCD\_RisC\_Lancet\_2020\_height\_child\_adolescent\_global.csv"}\NormalTok{) }

\CommentTok{\# or using read table, make sure you specify there\textquotesingle{}s a header!}
\NormalTok{data }\OtherTok{\textless{}{-}} \FunctionTok{read.table}\NormalTok{(}\StringTok{"../data/NCD\_RisC\_Lancet\_2020\_height\_child\_adolescent\_global.csv"}\NormalTok{, }\AttributeTok{sep =} \StringTok{","}\NormalTok{, }\AttributeTok{header=}\NormalTok{T)}
\end{Highlighting}
\end{Shaded}

\end{tcolorbox}

\begin{itemize}
\tightlist
\item
  How many data entries are there?
\end{itemize}

\begin{tcolorbox}[enhanced jigsaw, breakable, title=\textcolor{quarto-callout-tip-color}{\faLightbulb}\hspace{0.5em}{Code and answer}, toptitle=1mm, leftrule=.75mm, toprule=.15mm, colbacktitle=quarto-callout-tip-color!10!white, opacityback=0, opacitybacktitle=0.6, bottomtitle=1mm, arc=.35mm, titlerule=0mm, colframe=quarto-callout-tip-color-frame, left=2mm, bottomrule=.15mm, rightrule=.15mm, coltitle=black, colback=white]

\begin{Shaded}
\begin{Highlighting}[]
\FunctionTok{nrow}\NormalTok{(data) }
\end{Highlighting}
\end{Shaded}

1050 entries because there are 1050 rows, excluding the header

\end{tcolorbox}

\begin{itemize}
\tightlist
\item
  What is the youngest age group in the dataset, and what is the oldest
  age group (i.e.~how old are those groups)?
\end{itemize}

\begin{tcolorbox}[enhanced jigsaw, breakable, title=\textcolor{quarto-callout-tip-color}{\faLightbulb}\hspace{0.5em}{Code and answer}, toptitle=1mm, leftrule=.75mm, toprule=.15mm, colbacktitle=quarto-callout-tip-color!10!white, opacityback=0, opacitybacktitle=0.6, bottomtitle=1mm, arc=.35mm, titlerule=0mm, colframe=quarto-callout-tip-color-frame, left=2mm, bottomrule=.15mm, rightrule=.15mm, coltitle=black, colback=white]

First we need to know what columns there are in the data and what they
are called, which can be done in different ways.

Remember you can always view the data (by clicking on it in the
environment or by calling \texttt{View(data})).

\begin{Shaded}
\begin{Highlighting}[]
\CommentTok{\#this shows you just the structure of the table}
\FunctionTok{str}\NormalTok{(data) }

\CommentTok{\#this tells you the column names of the table}
\FunctionTok{names}\NormalTok{(data) }

\CommentTok{\# this shows you the first 6 entries and the column names}
\FunctionTok{head}\NormalTok{(data) }
\end{Highlighting}
\end{Shaded}

From this we can see that the column that stores the age is called
\texttt{Age.group}

\begin{Shaded}
\begin{Highlighting}[]
\CommentTok{\# minimum age i.e. youngest age class}
\FunctionTok{min}\NormalTok{(data}\SpecialCharTok{$}\NormalTok{Age.group) }

\CommentTok{\# maximum age i.e. oldest age class}
\FunctionTok{max}\NormalTok{(data}\SpecialCharTok{$}\NormalTok{Age.group) }
\end{Highlighting}
\end{Shaded}

So 5 and 19, respectively.

\end{tcolorbox}

\subsection{Part 2: cleaning the data}\label{part-2-cleaning-the-data}

\begin{itemize}
\tightlist
\item
  What is the data type / class of the mean height and its standard
  error?
\end{itemize}

\begin{tcolorbox}[enhanced jigsaw, breakable, title=\textcolor{quarto-callout-tip-color}{\faLightbulb}\hspace{0.5em}{Code and answer}, toptitle=1mm, leftrule=.75mm, toprule=.15mm, colbacktitle=quarto-callout-tip-color!10!white, opacityback=0, opacitybacktitle=0.6, bottomtitle=1mm, arc=.35mm, titlerule=0mm, colframe=quarto-callout-tip-color-frame, left=2mm, bottomrule=.15mm, rightrule=.15mm, coltitle=black, colback=white]

\begin{Shaded}
\begin{Highlighting}[]
\CommentTok{\# this tells us the class of a specific column}
\FunctionTok{class}\NormalTok{(data}\SpecialCharTok{$}\NormalTok{Mean.height) }
\FunctionTok{class}\NormalTok{(data}\SpecialCharTok{$}\NormalTok{Mean.height.standard.error) }

\CommentTok{\# this tells us the class of all columns}
\FunctionTok{str}\NormalTok{(data) }
\end{Highlighting}
\end{Shaded}

Both are characters/strings.

\end{tcolorbox}

\begin{itemize}
\tightlist
\item
  Hmm, this data class is probably incorrect\ldots{} It should be
  numeric, right? What's causing this?
\end{itemize}

\begin{tcolorbox}[enhanced jigsaw, breakable, title=\textcolor{quarto-callout-tip-color}{\faLightbulb}\hspace{0.5em}{Code and answer}, toptitle=1mm, leftrule=.75mm, toprule=.15mm, colbacktitle=quarto-callout-tip-color!10!white, opacityback=0, opacitybacktitle=0.6, bottomtitle=1mm, arc=.35mm, titlerule=0mm, colframe=quarto-callout-tip-color-frame, left=2mm, bottomrule=.15mm, rightrule=.15mm, coltitle=black, colback=white]

A classic coding issue, ``what's up with my data''? Figuring out why
something is not behaving as expected will happen most of the time you
are coding. How do we diagnose the problem?

We can start by having a look at the data using e.g.~\texttt{View(data)}
and by doing some tests.

\begin{Shaded}
\begin{Highlighting}[]
\CommentTok{\# let\textquotesingle{}s see what happens when we change the data class to numeric, and store this in a test variable.}

\NormalTok{test }\OtherTok{\textless{}{-}} \FunctionTok{as.numeric}\NormalTok{(data}\SpecialCharTok{$}\NormalTok{Mean.height)  }
\end{Highlighting}
\end{Shaded}

It gives a warning which indicates it could not change one or more of
the values to numeric so it had to be put to NA. Why?

What sometimes works is to sort the column, this might help us diagnose
what is happening in such a large table. When we look at
\texttt{View(data)} / the data tab, we can click on the column
\texttt{Mean.height} and it will sort it. Click on it again, and it will
sort it in reverse order

Whatever way you used, you might have identified the issue!

There are two ``ERRORS'' at row number 112, a boy of age class 1992 and
age 11. This text is causing the error, as R doesn't recognise this
column as numeric now.

\end{tcolorbox}

\begin{itemize}
\tightlist
\item
  Fix the error and change the data class to numeric
\end{itemize}

\begin{tcolorbox}[enhanced jigsaw, breakable, title=\textcolor{quarto-callout-tip-color}{\faLightbulb}\hspace{0.5em}{Code and answer}, toptitle=1mm, leftrule=.75mm, toprule=.15mm, colbacktitle=quarto-callout-tip-color!10!white, opacityback=0, opacitybacktitle=0.6, bottomtitle=1mm, arc=.35mm, titlerule=0mm, colframe=quarto-callout-tip-color-frame, left=2mm, bottomrule=.15mm, rightrule=.15mm, coltitle=black, colback=white]

We can't correct the value because we don't know what happened with this
boy's height measurement. So let's either remove the entire row or
change the values to NA. You could do this manually by going into the
raw data using e.g.~Excel. But a better practise is by doing this in the
code - this way we can always track exactly what we did, and others can
see it too because it is documented.

In the first method below, we simply remove the 112'th row, so we remove
the entire data point. It's always smart to store something in a
different variable if you remove something so you do not override the
original data, I'm showing you an example here but we'll continue with
the `data' variable as we know it's all good.

In the second and third method, we remove the text ``ERROR'' and instead
replace it with the `official' NA value that R recognises as NA.

\begin{Shaded}
\begin{Highlighting}[]
\DocumentationTok{\#\# method 1}
\CommentTok{\# remove the row}
\NormalTok{data\_clean }\OtherTok{\textless{}{-}}\NormalTok{ data[}\SpecialCharTok{{-}}\DecValTok{112}\NormalTok{,] }

\CommentTok{\# then we change the data class to numeric and we do not get a warning}
\NormalTok{data\_clean}\SpecialCharTok{$}\NormalTok{Mean.height }\OtherTok{\textless{}{-}} \FunctionTok{as.numeric}\NormalTok{(data\_clean}\SpecialCharTok{$}\NormalTok{Mean.height) }
\CommentTok{\# same for the SE}
\NormalTok{data\_clean}\SpecialCharTok{$}\NormalTok{Mean.height.standard.error }\OtherTok{\textless{}{-}} \FunctionTok{as.numeric}\NormalTok{(data\_clean}\SpecialCharTok{$}\NormalTok{Mean.height.standard.error) }

\DocumentationTok{\#\# method 2}
\CommentTok{\# this is what we had done before and automatically put the \textquotesingle{}character\textquotesingle{} ERROR as NA, you\textquotesingle{}ll get a warning but now we know why}

\NormalTok{data}\SpecialCharTok{$}\NormalTok{Mean.height }\OtherTok{\textless{}{-}} \FunctionTok{as.numeric}\NormalTok{(data}\SpecialCharTok{$}\NormalTok{Mean.height) }

\CommentTok{\# same for the SE}
\NormalTok{data}\SpecialCharTok{$}\NormalTok{Mean.height.standard.error }\OtherTok{\textless{}{-}} \FunctionTok{as.numeric}\NormalTok{(data}\SpecialCharTok{$}\NormalTok{Mean.height.standard.error) }

\DocumentationTok{\#\# method 3}

\CommentTok{\# this is the same as above, but manually which is a bit longer and more complex}

\NormalTok{data}\SpecialCharTok{$}\NormalTok{Mean.height[}\FunctionTok{which}\NormalTok{(data}\SpecialCharTok{$}\NormalTok{Mean.height}\SpecialCharTok{==}\StringTok{"ERROR"}\NormalTok{)] }\OtherTok{\textless{}{-}} \ConstantTok{NA} 
\NormalTok{data}\SpecialCharTok{$}\NormalTok{Mean.height.standard.error[}\FunctionTok{which}\NormalTok{(data}\SpecialCharTok{$}\NormalTok{Mean.height.standard.error}\SpecialCharTok{==}\StringTok{"ERROR"}\NormalTok{)] }\OtherTok{\textless{}{-}} \ConstantTok{NA} 

\CommentTok{\# change to numeric, now we won\textquotesingle{}t get a warning}
\NormalTok{data}\SpecialCharTok{$}\NormalTok{Mean.height }\OtherTok{\textless{}{-}} \FunctionTok{as.numeric}\NormalTok{(data}\SpecialCharTok{$}\NormalTok{Mean.height) }
\NormalTok{data}\SpecialCharTok{$}\NormalTok{Mean.height.standard.error }\OtherTok{\textless{}{-}} \FunctionTok{as.numeric}\NormalTok{(data}\SpecialCharTok{$}\NormalTok{Mean.height.standard.error) }
\end{Highlighting}
\end{Shaded}

There we go! It's always good to check if things worked, so you can use
e.g.~\texttt{str(data)} now to check that R indeed changed the columns
to numeric.

\end{tcolorbox}

\subsection{Part 3: summary statistics}\label{part-3-summary-statistics}

\begin{itemize}
\tightlist
\item
  What is the mean across years of the mean heights in the dataset?
\end{itemize}

\begin{tcolorbox}[enhanced jigsaw, breakable, title=\textcolor{quarto-callout-tip-color}{\faLightbulb}\hspace{0.5em}{Code and answer}, toptitle=1mm, leftrule=.75mm, toprule=.15mm, colbacktitle=quarto-callout-tip-color!10!white, opacityback=0, opacitybacktitle=0.6, bottomtitle=1mm, arc=.35mm, titlerule=0mm, colframe=quarto-callout-tip-color-frame, left=2mm, bottomrule=.15mm, rightrule=.15mm, coltitle=black, colback=white]

The function that calculates means is, intuitively, \texttt{mean()}
which is a base function (i.e.~you don't need to install a package for
this). If you want to know more details about how to use a function, you
can use \texttt{?mean}

\begin{Shaded}
\begin{Highlighting}[]
\CommentTok{\# we have to make sure we remove the NA when calculating the mean though!}
\FunctionTok{mean}\NormalTok{(data}\SpecialCharTok{$}\NormalTok{Mean.height, }\AttributeTok{na.rm =}\NormalTok{ T) }

\CommentTok{\# or we use the data that doesn\textquotesingle{}t contain the NA}
\FunctionTok{mean}\NormalTok{(data\_clean}\SpecialCharTok{$}\NormalTok{Mean.height) }
\end{Highlighting}
\end{Shaded}

\end{tcolorbox}

\begin{itemize}
\tightlist
\item
  What is the mean across years of the mean height, calculated
  separately for boys and girls?
\end{itemize}

\begin{tcolorbox}[enhanced jigsaw, breakable, title=\textcolor{quarto-callout-tip-color}{\faLightbulb}\hspace{0.5em}{Code and answer}, toptitle=1mm, leftrule=.75mm, toprule=.15mm, colbacktitle=quarto-callout-tip-color!10!white, opacityback=0, opacitybacktitle=0.6, bottomtitle=1mm, arc=.35mm, titlerule=0mm, colframe=quarto-callout-tip-color-frame, left=2mm, bottomrule=.15mm, rightrule=.15mm, coltitle=black, colback=white]

Here's where your preference can kick in: do you want to use dplyr or
something else, like base R or data.table? I will show you multiple
options, just see what looks more intuitive to you. To make sure we can
use dplyr, we have to load the library.

To answer this question, we have to do multiple steps: divide the data
among the sexes somehow, and calculate the mean. This can be done in
steps, or in one go. If you're a computational scientists, you would
want things to be done in as little code as possible. But since we are
learning, just do what makes most sense to you.

\begin{Shaded}
\begin{Highlighting}[]
\CommentTok{\# method 1 with dplyr}
\FunctionTok{library}\NormalTok{(dplyr) }\CommentTok{\# we can use dplyr to use the filter option}

\NormalTok{girls }\OtherTok{\textless{}{-}}\NormalTok{ data }\SpecialCharTok{\%\textgreater{}\%} \FunctionTok{filter}\NormalTok{(Sex }\SpecialCharTok{==} \StringTok{"Girls"}\NormalTok{)}
\NormalTok{boys }\OtherTok{\textless{}{-}}\NormalTok{ data }\SpecialCharTok{\%\textgreater{}\%} \FunctionTok{filter}\NormalTok{(Sex }\SpecialCharTok{==} \StringTok{"Boys"}\NormalTok{)}

\FunctionTok{mean}\NormalTok{(girls}\SpecialCharTok{$}\NormalTok{Mean.height, }\AttributeTok{na.rm =}\NormalTok{ T) }
\FunctionTok{mean}\NormalTok{(boys}\SpecialCharTok{$}\NormalTok{Mean.height, }\AttributeTok{na.rm =}\NormalTok{ T) }

\CommentTok{\# method 2 without dplyr}
\FunctionTok{mean}\NormalTok{(data}\SpecialCharTok{$}\NormalTok{Mean.height[}\FunctionTok{which}\NormalTok{(data}\SpecialCharTok{$}\NormalTok{Sex }\SpecialCharTok{==} \StringTok{"Girls"}\NormalTok{)], }\AttributeTok{na.rm=}\NormalTok{T) }
\FunctionTok{mean}\NormalTok{(data}\SpecialCharTok{$}\NormalTok{Mean.height[}\FunctionTok{which}\NormalTok{(data}\SpecialCharTok{$}\NormalTok{Sex }\SpecialCharTok{==} \StringTok{"Boys"}\NormalTok{)], }\AttributeTok{na.rm=}\NormalTok{T) }
\end{Highlighting}
\end{Shaded}

\end{tcolorbox}

\begin{itemize}
\tightlist
\item
  In what year was the standard error lowest for boys and girls
  separately, for the age class 12? (i.e.~the variance smallest?)
\end{itemize}

Again, we need to take multiple steps here: only look at data from the
Age group 12, and then look at what year the variance was the smallest
(i.e.~the minimal variance).

I'll only show you the dplyr way, but you can use many different methods
to get to the same answer here.

\begin{tcolorbox}[enhanced jigsaw, breakable, title=\textcolor{quarto-callout-tip-color}{\faLightbulb}\hspace{0.5em}{Code and answer}, toptitle=1mm, leftrule=.75mm, toprule=.15mm, colbacktitle=quarto-callout-tip-color!10!white, opacityback=0, opacitybacktitle=0.6, bottomtitle=1mm, arc=.35mm, titlerule=0mm, colframe=quarto-callout-tip-color-frame, left=2mm, bottomrule=.15mm, rightrule=.15mm, coltitle=black, colback=white]

\begin{Shaded}
\begin{Highlighting}[]
\CommentTok{\# method 1: with dplyr}
\NormalTok{girls\_12yr }\OtherTok{\textless{}{-}}\NormalTok{ girls }\SpecialCharTok{\%\textgreater{}\%} \FunctionTok{filter}\NormalTok{(Age.group }\SpecialCharTok{==} \DecValTok{12}\NormalTok{)}
\NormalTok{boys\_12yr }\OtherTok{\textless{}{-}}\NormalTok{ boys }\SpecialCharTok{\%\textgreater{}\%} \FunctionTok{filter}\NormalTok{(Age.group }\SpecialCharTok{==} \DecValTok{12}\NormalTok{)}

\NormalTok{girls\_12yr }\SpecialCharTok{\%\textgreater{}\%} \FunctionTok{arrange}\NormalTok{(Mean.height.standard.error) }\SpecialCharTok{\%\textgreater{}\%} \FunctionTok{head}\NormalTok{() }\CommentTok{\# 2003, with a SE of 0.09367336}

\NormalTok{boys\_12yr }\SpecialCharTok{\%\textgreater{}\%} \FunctionTok{arrange}\NormalTok{(Mean.height.standard.error) }\SpecialCharTok{\%\textgreater{}\%} \FunctionTok{head}\NormalTok{() }\CommentTok{\# 2004, with a SE of 0.1342430}
\end{Highlighting}
\end{Shaded}

\end{tcolorbox}

\subsection{Part 4: plotting}\label{part-4-plotting}

\begin{itemize}
\tightlist
\item
  Make a plot that shows you whether the mean height of 18 year olds
  increased or decreased over time. Bonus if you do it seperately per
  sex!
\end{itemize}

\begin{tcolorbox}[enhanced jigsaw, breakable, title=\textcolor{quarto-callout-tip-color}{\faLightbulb}\hspace{0.5em}{Code and answer}, toptitle=1mm, leftrule=.75mm, toprule=.15mm, colbacktitle=quarto-callout-tip-color!10!white, opacityback=0, opacitybacktitle=0.6, bottomtitle=1mm, arc=.35mm, titlerule=0mm, colframe=quarto-callout-tip-color-frame, left=2mm, bottomrule=.15mm, rightrule=.15mm, coltitle=black, colback=white]

Let's get plotting! First choose: do you want to use ggplot2
(recommended) or base R to make your plot?

Before you start making code, think what elements should be on the axes.
What should be on the X axis, what variable on the Y axis, and what type
of graph would represent the data best? A boxplot, scatterplot, line
graph, pie chart? Choose the right element accordingly and then start
writing the code.

We also have to use a portion of the data only, again. We can use dplyr
(\texttt{filter()}) but I will also show you another function here,
\texttt{subset()} which is a base R function to do something similar to
\texttt{filter()}.

\begin{Shaded}
\begin{Highlighting}[]
\CommentTok{\# method 1: with ggplot}
\FunctionTok{library}\NormalTok{(ggplot2)}

\FunctionTok{ggplot}\NormalTok{(}\FunctionTok{subset}\NormalTok{(data, Age.group }\SpecialCharTok{==} \DecValTok{18}\NormalTok{), }\FunctionTok{aes}\NormalTok{(}\AttributeTok{x =}\NormalTok{ Year, }\AttributeTok{y =}\NormalTok{ Mean.height)) }\SpecialCharTok{+} \FunctionTok{geom\_point}\NormalTok{(}\FunctionTok{aes}\NormalTok{(}\AttributeTok{col =}\NormalTok{ Sex)) }\SpecialCharTok{+} \FunctionTok{geom\_line}\NormalTok{(}\FunctionTok{aes}\NormalTok{(}\AttributeTok{col =}\NormalTok{ Sex)) }\SpecialCharTok{+} \FunctionTok{theme\_classic}\NormalTok{()}

\CommentTok{\# method 2: base R}
\CommentTok{\# subset the data}
\NormalTok{age\_18 }\OtherTok{\textless{}{-}} \FunctionTok{subset}\NormalTok{(data, Age.group }\SpecialCharTok{==} \DecValTok{18}\NormalTok{)}

\CommentTok{\# set up empty plot}
\FunctionTok{plot}\NormalTok{(}
\NormalTok{  age\_18}\SpecialCharTok{$}\NormalTok{Year, age\_18}\SpecialCharTok{$}\NormalTok{Mean.height,}
  \AttributeTok{type =} \StringTok{"n"}\NormalTok{,}
  \AttributeTok{xlab =} \StringTok{"Year"}\NormalTok{,}
  \AttributeTok{ylab =} \StringTok{"Mean height"}
\NormalTok{)}

\CommentTok{\# add points and lines by Sex}
\ControlFlowTok{for}\NormalTok{ (s }\ControlFlowTok{in} \FunctionTok{levels}\NormalTok{(age\_18}\SpecialCharTok{$}\NormalTok{Sex)) \{}
\NormalTok{  ds }\OtherTok{\textless{}{-}}\NormalTok{ age\_18[age\_18}\SpecialCharTok{$}\NormalTok{Sex }\SpecialCharTok{==}\NormalTok{ s, ]}
  
  \FunctionTok{points}\NormalTok{(ds}\SpecialCharTok{$}\NormalTok{Year, ds}\SpecialCharTok{$}\NormalTok{Mean.height,}
         \AttributeTok{col =}\NormalTok{ cols[s], }\AttributeTok{pch =} \DecValTok{16}\NormalTok{)}
  
  \FunctionTok{lines}\NormalTok{(ds}\SpecialCharTok{$}\NormalTok{Year, ds}\SpecialCharTok{$}\NormalTok{Mean.height,}
        \AttributeTok{col =}\NormalTok{ cols[s], }\AttributeTok{lwd =} \DecValTok{2}\NormalTok{)}
\NormalTok{\}}

\CommentTok{\# add legend}
\FunctionTok{legend}\NormalTok{(}\StringTok{"topleft"}\NormalTok{,}
       \AttributeTok{legend =} \FunctionTok{levels}\NormalTok{(}\FunctionTok{as.factor}\NormalTok{(age\_18}\SpecialCharTok{$}\NormalTok{Sex)),}
       \AttributeTok{pch =} \DecValTok{16}\NormalTok{,}
       \AttributeTok{lwd =} \DecValTok{2}\NormalTok{,}
       \AttributeTok{bty =} \StringTok{"n"}\NormalTok{)}
\end{Highlighting}
\end{Shaded}

As we see, both the mean height of girls and boys aged 18 years
increased over time. What is also clear is that base R is a lot longer,
and you require a lot of separate elements.

\end{tcolorbox}

Well done! If you managed to answer all of the above, with or without
some help from your notes and the internet, you're ready to explore
genetic data and answer some pop gen questions.




\end{document}
